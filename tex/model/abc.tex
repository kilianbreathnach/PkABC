\documentclass[12pt, preprint]{aastex}
\usepackage{graphicx}	% For figures
\usepackage{natbib}	% For citep and citep
\usepackage{amsmath}	% for \iint
\usepackage{bbm}
\usepackage[breaklinks]{hyperref}	% for blackboard bold numbers
\usepackage{hyperref}
\hypersetup{colorlinks}
\usepackage{color}
\usepackage{morefloats}
\definecolor{darkred}{rgb}{0.5,0,0}
\definecolor{darkgreen}{rgb}{0,0.5,0}
\definecolor{darkblue}{rgb}{0,0,0.5}
\hypersetup{ colorlinks,
linkcolor=darkblue,
filecolor=darkgreen,
urlcolor=darkred,
citecolor=darkblue }

\DeclareMathOperator*{\argmax}{arg\,max}

\newcommand{\beq}{\begin{equation}}
\newcommand{\eeq}{\end{equation}}

\begin{document}

\author{
  Mohammadjavad~Vakili\altaffilmark{1},
  ChangHoon~Hahn\altaffilmark{1}, ...}
\altaffiltext{1}{Center for Cosmology and Particle Physics, Department of Phyics,
             New York University, 4 Washington Pl., room 424, New York, NY, 10003, USA}

\email{mjvakili@nyu.edu}

\title{Likelihood free inference of cosmological parameters from measurements of galaxy clustering in SDSS III BOSS data}

\begin{abstract}

Measurement of the large scale clustering of galaxies has been 
one of the most crucial methods in constraining cosmological 
parameters and studying the expansion history of the universe.
Numerous studies have been carried out on studying the clustering 
of galaxies in \emph{Sloan Digital Sky Survey} Baryonic Oscillation 
Spectroscopic Survey, \emph{BOSS}, in order to put constraint on 
cosmological parameters, and to test gravity on large scales.
These constraints are achived by performing Bayesian analysis, which 
relies on evaluating the likelihood function, the probability of 
the observing the data given the model. Likelihood evaluation however, 
is computationally expensive, and requires making assumption about the 
form of the function, usually assumed Gaussian distribution. 
In this work we apply the Approximate Bayesian Computation method, ABC, 
to the problem of infering cosmological parameters from measurements of 
the galaxy power spectrum in SDSS BOSS data. This method requires a prior over 
cosmological parameters from which large set of parameters are sampled from, 
forward modelling of the galaxy-galaxy power spectrum which takes cosmological 
parameters as input, measurement of galaxy power spectrum, and a distance metric 
for comparison of the measured power spectrum with the model.
In this work, the model of galaxy power spectrum is computed by 2-loop renormalized 
perturbation theory, convolved with survey window function. Furthermore, an 
adaptive importance sampling extension of ABC is used in order to make the sampling 
more efficient. 

\end{abstract}

\section{Introduction}

Parameter estimation in cosomlogy is commonly done with Bayesian inference. 
In the Bayesian framework, constraints on cosmological parameters are achieved by 
sampling from the posterior distribution, the probability of the parameters of the 
model given the data. Posteriors are evaluated by multiplying the prior probability 
distribution of the paramters of the model with the likelihood function, the probability 
of the observed data given the parameters of the model.
Therefore, Bayesian inference relies on the evaluation of the likelihood function.
 
Evaluation of the likelihood requires on making assumption about the functional 
form of the likelihood. It is often assumed that the likelihood is Gaussian distribution 
with a given covariance matrix that needs to be evaluated from simulating a large set of 
mock catalogs, or other techniques such as bootstrap. In order to 
compute the likelihood funciton, one needs to invert a high dimensional covariance 
matrix which makes the likelihood evaluation and therefore sampling from the posterior 
computationally intractable.

Approximate Bayesian Computation, ABC, is a method that eliminates the need for likelihood. 
In recent years, this technique has been used in a number of cosmological studies, such as 
the inference of cosmological parameters from modelling of the distances of supernovae type Ia, 
and modelling of weak lensing peak maps. Exploring the parameter space with ABC algorithm 
requires three essential ingredients: (1) a prior pdf on the parameters that we wish to estimate, 
(2) a forward model (simulator) of the observations, and (3) a distance metric for comparing 
the model and the observed data. In problems where likelihood evaluation is not analytically or 
computationally tractable, this algorithm bypasses the need for likelihood evaluation.

To date, different variations of ABC algorithm have been introduced  and implemented 
in the literature. One of the earliest methods is likelihood free Markov Chain Monte 
Carlo. In this algorithm, an initial draw from the prior is kept, if the distance between 
the simulated model corresponding to that parameter and the observed data is less than a threshold. 
In the consequent steps, the proposed parameters are accepted if \emph{first} they satisfy the 
distance requirement, and \emph{second} the Metropolis-Hasting (MH) acceptance ratio evaluated for the 
proposal is less than one. In the MH acceptance ratio, the likelihood is replaced with some transition 
kernel. 

Based on importance sampling, several modifications to this algorithm have been proposed. 
These methods are called ABC Sequential Monte Carlo (ABC-SMC), or ABC Population Monte Carlo 
(ABC-PMC). These algorithms rely on sampling a large set of parameters from the prior pdf, and assigning 
equal weights to them as an initial step. In the subsequent steps, sampling is modified by 
proposing intermediate distribution built from the previous sample of parameters and their 
corresponding weights. Parameters are updated by sampling from the intermediate distribution 
and are accepted if they pass the distance requirements. Furthermore, weights are updated according 
to the intermediate distribution. 

Talk about importance sampling here ...
Sampling from the parameter space in ABC algorithm can become more efficient by 
employing adaptive importance sampling... 

\section{Method}

\subsection{Approximate Bayesian Computation}

\subsubsection{basics}

Talk about generic abc ...

\subsubsection{ABC with adaptive importance sampling}



\subsection{Model of galaxy power spectrum}


\subsubsection{Renormalized perturbation theory}

\section{Data}

\subsection{SDSS BOSS}

\subsection{Measurements of galaxy-galaxy power spectrum}

\section{Analysis}

\section{Results}


\begin{thebibliography}{70}

\bibitem[Bertin \& Arnouts (1996)]{sex} Bertin, E., Arnouts, S. \ 1996  Astronomy and Astrophysics Supplement Series, 117, 393-40
\bibitem[Cram\'{e}r (1946)]{cramer} Cram\'{e}r, H. \ 1946 Mathematical methods of statistics, Princeton university press
\bibitem[Le Cam (1953)]{lecam} Le Cam, L. M. \ 1953 On some asymptotic properties of maximum likelihood estimates and related Bayes' estimates (Vol. 1, No. 11), University of California press
\bibitem[Lupton etal. (2001)]{sdss} Lupton, R. etal. \ 2001  arXiv: astro-ph/0101420
\bibitem[Schechter etal. (1993)]{dophot} Schechter, P. L., Mateo, M., Saha, A. \ 1993 \pasp, 1342-1353
\bibitem[Stetson (1987)]{daophot} Stetson, P. B. \ 1987, \pasp, 191-222
\bibitem[Trujillo etal. (2001)]{moffat} Trujillo, I., et al. \ 2001 \mnras, 977-985

\end{thebibliography}


\end{document}
