\documentclass{article}

%define general packages
\usepackage{fullpage}
\usepackage{epsfig}
\usepackage{amsmath}
\usepackage{natbib}
\usepackage{footnote}
\usepackage{courier}

%internal short cuts
\def \HgA {H$\gamma_A$}
\def \gon {Gonz\'{a}lez}
\def \Hbp {H$\beta ^\prime$}

\begin{document}

\title{Observed BOSS Galaxy Powerspectrum}
\author{ChangHoon Hahn, Mohammadjavad Vakili, Kilian Walsh}

\begin{abstract}
Observed galaxy powerspectrum is necessary to fit and calibrate the parameters 
of the halo model powerspectrum. To calculate the observed powerspectrum we use 
the Baryon Oscillation Spectroscopic Survey (BOSS) CMASS and LOWZ Data Release 12 
combined sample, which consists of $850,364$ galaxies over the redshift range 
$0.2 < z < 0.75$. We divide our sample into redshift bins of $\Delta z = $ to 
maximize the number of redshift bins while keeping that the effects of cosmic 
variance below (quote some number). 
% Motivate the reason why we want more redshift bins

Furthermore, in calcuating the galaxy powerspectrum we use 
the fiber collision correction method described in Hahn et al. (in prep), which 
accounts for systematic effects in the shot noise correction term of the 
powerspectrum estimator and statistically reconstructs the small scale clustering 
by modeling the line-of-sight displacement of fiber collided pairs. Using the
correction method, we reduce the effect of fiber collisions to $<\% 1$ over the 
$k < 0.82\;h/\mathrm{Mpc}$. Combining the cosmic variance estimates from simulated
mock catalogs with uncertainty contributions from systematic effects, we present
observed powerspectrum measurements with uncertainties of $< \mathrm{somenumber}$. 
\end{abstract}

\section{BOSS Galaxy Sample} \label{sec:cmasslowz}

\begin{itemize}
\item Describe CMASS LOWZ combined sample DR 12 
\item Describe systematic effects here 
\end{itemize}

\section{Powerspectrum}
\subsection{Estimator} 
\begin{itemize}
\item Describe revised FKP estimator to account for systematics in BOSS galaxy sample 
\end{itemize}

\subsection{Fiber Collision Correction}
\begin{itemize}
\item Describe, briefly, the fiber collision correction method of Hahn et al. (in prep.) 
\item Present fiber collision corrected P(k) for entire combined sample? just CMASS? two bin combined sample? 
\end{itemize}

\subsection{Redshift Binning} 
\begin{itemize}
\item Describe the motivation for narrow redshift binning 
\item Describe observational challenges of narrow redshift binning
\item Descirbe final redshift binning and cosmic variance modeling using nseries mocks 
\end{itemize}

\end{document}
