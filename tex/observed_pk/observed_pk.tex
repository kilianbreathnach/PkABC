\documentclass{article}

%define general packages
\usepackage{epsfig}
\usepackage{amsmath}
\usepackage{natbib}
\usepackage{footnote}
\usepackage{courier}

%internal short cuts
\def \HgA {H$\gamma_A$}
\def \gon {Gonz\'{a}lez}
\def \Hbp {H$\beta ^\prime$}

\begin{document}

\title{Observed BOSS Galaxy Powerspectrum}
\author{ChangHoon Hahn, Mohammadjavad Vakili, Kilian Walsh}

\begin{abstract}
In order to We describe the calculation of the observed SDSS-III BOSS galaxy powerspectrum used to compare the halo model powerspectrum prediction. 
\end{abstract}

\section{Sample Selection}
\subsection{BOSS Galaxy Sample} \label{sec:cmasslowz}
\begin{itemize}
\item Describe CMASS LOWZ combined sample here 
\item Describe systematic effects here 
\end{itemize}

\section{Powerspectrum}
\subsection{Estimator} 
\begin{itemize}
\item Describe revised FKP estimator to account for systematics in BOSS galaxy sample 
\end{itemize}

\subsection{Fiber Collision Correction}
\begin{itemize}
\item Describe, briefly, the fiber collision correction method of Hahn et al. (in prep.) 
\item Present fiber collision corrected P(k) for entire combined sample? just CMASS? two bin combined sample? 
\end{itemize}

\subsection{Redshift Binning} 
\begin{itemize}
\item Describe the motivation for narrow redshift binning 
\item Describe observational challenges of narrow redshift binning
\item Descirbe final redshift binning and cosmic variance modeling using nseries mocks 
\end{itemize}

\end{document}
